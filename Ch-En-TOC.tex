\documentclass[openany]{book}
\usepackage[UTF8,heading=true]{ctex}
\usepackage{url}
\usepackage[linkcolor=blue,
            CJKbookmarks=true,
            citecolor=blue,
            urlcolor=blue,
            colorlinks,
            pdfauthor={Marlin, <marlinhz@foxmail.com>}]
            {hyperref} %% pdf 结果文档属性.

\begin{document}
\makeatletter
\newcommand\engcontentsname{Contents}
\newcommand\tableofengcontents{%
	\if@twocolumn
	  \@restonecoltrue\onecolumn
	\else
	  \@restonecolfalse
	\fi
    \chapter*{\engcontentsname
		\@mkboth{%
			\MakeUppercase\engcontentsname}{\MakeUppercase\engcontentsname}}%
	\@starttoc{toe}% !!!!Define a new contents!!!!
	\if@restonecol\twocolumn\fi
}
\newcommand\addengcontents[2]{%
	\addcontentsline{toe}{#1}{\protect\numberline{\csname the#1\endcsname}#2}}
\makeatother

\newcommand\echapter[1]{\addengcontents{chapter}{#1}}
\newcommand\esection[1]{\addengcontents{section}{#1}}
\newcommand\esubsection[1]{\addengcontents{subsection}{#1}}

\title{中英文目录测试}
\author{Marlin}
\maketitle
%\frontmatter

\tableofcontents % 中文目录
\tableofengcontents % 英文目录

\chapter{贾森·基德}
\echapter{Jason Kidd}
要是没有基德,网队会成什么样?这个问题就跟马刺队没有邓肯一样,恐怕什么都不是,
什么气候都成不了。
\chapter{蒂姆·邓肯}
\echapter{Tim Duncan}
北京时间7日进行的NBA总决赛第二场,网队赢了,主要赢在基德,而马刺输了,也主要输
在邓肯。他19分、12个篮板、4个盖帽的表现,虽然比第一场32分、20个篮板和7个盖帽的
表现逊色,但并不算糟糕。问题是,他的罚球太差了,太有失MVP称号。他获得了10次罚球
机会,只罚中可怜巴巴的3个球。最后马刺以85比87败北,邓肯是有责任的。如果他罚篮的
命中率再稍稍高一点儿,哪怕达到50\%,马刺也就赢了。赛后邓肯也两眼发直,低头不语,
一脸怒气。与邓肯相比,基德光芒四射,以30分、7个篮板的表现,走出低迷,报了私仇,
也报了公仇。首场总决赛,基德仅得10分,网队以12分之差败北,如今在马刺主场战成1比
1平局,网队抢夺总冠军的野心再次膨胀,并为双方以后的争夺平添了悬念。
\section{新泽西网队}
\esection{New Jersey Nets}
基德的风头盖过了邓肯,除了得分以外,主要是罚球。他8次罚球6次命中,其中在终场前
1分多钟内就有6次罚球机会,结果6罚5中。当时,双方比分接近,马刺又追得很紧,如果
基德的罚球也像邓肯那么糟,网队也就完了。但是,邓肯的不足恰好是基德的长处,本场
比赛也不例外。基德复仇成功,也倾注了主教练斯科特的心血。他根据场上局势,不断变化
攻防手段,调整出场阵容。2.18米的老将穆托姆博得以再次出场,而且长达20分钟,就是
这位黑人教练的精心安排,而且这一招多少使马刺有些措手不及。这是他自5月9日以后的
第二次亮相,总决赛第一场,他上了3分钟,虽然没有得分,但给邓肯的那一记“火锅”,
以及倒地抢到篮板后的聪明叫停,给人印象极深。7日,斯科特给了他更多的上场时间,让
他凭借身高优势,遏制“双塔”,多抢篮板,进攻时让他尽量牵制对手。有了第一场的热
身和激情,穆托姆博的第二场亮相似乎注定比第一场更好。人们看到,他不仅得分了(4分),
而且篮板和盖帽分别增加到4个和3个。基德的风头盖过了邓肯,除了得分以外,主要是罚
球。他8次罚球6次命中,其中在终场前1分多钟内就有6次罚球机会,结果6罚5中。当时,
双方比分接近,马刺又追得很紧,如果基德的罚球也像邓肯那么糟,网队也就完了。但是,
邓肯的不足恰好是基德的长处,本场比赛也不例外。基德复仇成功,也倾注了主教练斯科
特的心血。他根据场上局势,不断变化攻防手段,调整出场阵容。2.18米的老将穆托姆博
得以再次出场,而且长达20分钟,就是这位黑人教练的精心安排,而且这一招多少使马刺
有些措手不及。这是他自5月9日以后的第二次亮相,总决赛第一场,他上了3分钟,虽然
没有得分,但给邓肯的那一记“火锅”,以及倒地抢到篮板后的聪明叫停,给人印象极深。
7日,斯科特给了他更多的上场时间,让他凭借身高优势,遏制“双塔”,多抢篮板,进
攻时让他尽量牵制对手。有了第一场的热身和激情,穆托姆博的第二场亮相似乎注定比第
一场更好。人们看到,他不仅得分了(4分),而且篮板和盖帽分别增加到4个和3个。基德
的风头盖过了邓肯,除了得分以外,主要是罚球。他8次罚球6次命中,其中在终场前1分
多钟内就有6次罚球机会,结果6罚5中。当时,双方比分接近,马刺又追得很紧,如果基
德的罚球也像邓肯那么糟,网队也就完了。但是,邓肯的不足恰好是基德的长处,本场比
赛也不例外。基德复仇成功,也倾注了主教练斯科特的心血。他根据场上局势,不断变化
攻防手段,调整出场阵容。2.18米的老将穆托姆博得以再次出场,而且长达20分钟,就是
这位黑人教练的精心安排,而且这一招多少使马刺有些措手不及。这是他自5月9日以后的
第二次亮相,总决赛第一场,他上了3分钟,虽然没有得分,但给邓肯的那一记“火锅”,
以及倒地抢到篮板后的聪明叫停,给人印象极深。7日,斯科特给了他更多的上场时间,让
他凭借身高优势,遏制“双塔”,多抢篮板,进攻时让他尽量牵制对手。有了第一场的热
身和激情,穆托姆博的第二场亮相似乎注定比第一场更好。人们看到,他不仅得分了(4分),
而且篮板和盖帽分别增加到4个和3个。基德的风头盖过了邓肯,除了得分以外,主要是罚
球。他8次罚球6次命中,其中在终场前1分多钟内就有6次罚球机会,结果6罚5中。当时,
双方比分接近,马刺又追得很紧,如果基德的罚球也像邓肯那么糟,网队也就完了。但是,
邓肯的不足恰好是基德的长处,本场比赛也不例外。基德复仇成功,也倾注了主教练斯科特
的心血。他根据场上局势,不断变化攻防手段,调整出场阵容。2.18米的老将穆托姆博得
以再次出场,而且长达20分钟,就是这位黑人教练的精心安排,而且这一招多少使马刺有些
措手不及。这是他自5月9日以后的第二次亮相,总决赛第一场,他上了3分钟,虽然没有得
分,但给邓肯的那一记“火锅”,以及倒地抢到篮板后的聪明叫停,给人印象极深。
\section{圣安东尼奥马刺队}
\esection{San Antonio Spurs}
7日,斯科特给了他更多的上场时间,让他凭借身高优势,遏制“双塔”,多抢篮板,进攻时
让他尽量牵制对手。有了第一场的热身和激情,穆托姆博的第二场亮相似乎注定比第一场更
好。人们看到,他不仅得分了(4分),而且篮板和盖帽分别增加到4个和3个。基德的风头盖
过了邓肯,除了得分以外,主要是罚球。他8次罚球6次命中,其中在终场前1分多钟内就有6
次罚球机会,结果6罚5中。当时,双方比分接近,马刺又追得很紧,如果基德的罚球也像邓
肯那么糟,网队也就完了。但是,邓肯的不足恰好是基德的长处,本场比赛也不例外。基德
复仇成功,也倾注了主教练斯科特的心血。他根据场上局势,不断变化攻防手段,调整出场
阵容。2.18米的老将穆托姆博得以再次出场,而且长达20分钟,就是这位黑人教练的精心安
排,而且这一招多少使马刺有些措手不及。这是他自5月9日以后的第二次亮相,总决赛第一
场,他上了3分钟,虽然没有得分,但给邓肯的那一记“火锅”,以及倒地抢到篮板后的聪明
叫停,给人印象极深。7日,斯科特给了他更多的上场时间,让他凭借身高优势,遏制“双塔”,
多抢篮板,进攻时让他尽量牵制对手。有了第一场的热身和激情,穆托姆博的第二场亮相似乎
注定比第一场更好。人们看到,他不仅得分了(4分),而且篮板和盖帽分别增加到4个和3个。
基德的风头盖过了邓肯,除了得分以外,主要是罚球。他8次罚球6次命中,其中在终场前1分
多钟内就有6次罚球机会,结果6罚5中。当时,双方比分接近,马刺又追得很紧,如果基德的
罚球也像邓肯那么糟,网队也就完了。但是,邓肯的不足恰好是基德的长处,本场比赛也不例
外。基德复仇成功,也倾注了主教练斯科特的心血。他根据场上局势,不断变化攻防手段,调
整出场阵容。2.18米的老将穆托姆博得以再次出场,而且长达20分钟,就是这位黑人教练的
精心安排,而且这一招多少使马刺有些措手不及。这是他自5月9日以后的第二次亮相,总决赛
第一场,他上了3分钟,虽然没有得分,但给邓肯的那一记“火锅”,以及倒地抢到篮板后的
聪明叫停,给人印象极深。
\subsection{巴特尔}
\esubsection{Mengke Bateer}
7日,斯科特给了他更多的上场时间,让他凭借身高优势,遏制“双塔”,多抢篮板,进攻时
让他尽量牵制对手。有了第一场的热身和激情,穆托姆博的第二场亮相似乎注定比第一场更好。
人们看到,他不仅得分了(4分),而且篮板和盖帽分别增加到4个和3个。基德的风头盖过了邓
肯,除了得分以外,主要是罚球。他8次罚球6次命中,其中在终场前1分多钟内就有6次罚球
机会,结果6罚5中。当时,双方比分接近,马刺又追得很紧,如果基德的罚球也像邓肯那么
糟,网队也就完了。但是,邓肯的不足恰好是基德的长处,本场比赛也不例外。基德复仇成功,
也倾注了主教练斯科特的心血。他根据场上局势,不断变化攻防手段,调整出场阵容。2.18米
的老将穆托姆博得以再次出场,而且长达20分钟,就是这位黑人教练的精心安排,而且这一招
多少使马刺有些措手不及。这是他自5月9日以后的第二次亮相,总决赛第一场,他上了3分钟,
虽然没有得分,但给邓肯的那一记“火锅”,以及倒地抢到篮板后的聪明叫停,给人印象极深。
7日,斯科特给了他更多的上场时间,让他凭借身高优势,遏制“双塔”,多抢篮板,进攻时
让他尽量牵制对手。有了第一场的热身和激情,穆托姆博的第二场亮相似乎注定比第一场更好。
人们看到,他不仅得分了(4分),而且篮板和盖帽分别增加到4个和3个。基德的风头盖过了邓
肯,除了得分以外,主要是罚球。他8次罚球6次命中,其中在终场前1分多钟内就有6次罚球机
会,结果6罚5中。当时,双方比分接近,马刺又追得很紧,如果基德的罚球也像邓肯那么糟,
网队也就完了。但是,邓肯的不足恰好是基德的长处,本场比赛也不例外。基德复仇成功,也
倾注了主教练斯科特的心血。他根据场上局势,不断变化攻防手段,调整出场阵容。2.18米的
老将穆托姆博得以再次出场,而且长达20分钟,就是这位黑人教练的精心安排,而且这一招多
少使马刺有些措手不及。这是他自5月9日以后的第二次亮相,总决赛第一场,他上了3分钟,虽
然没有得分,但给邓肯的那一记“火锅”,以及倒地抢到篮板后的聪明叫停,给人印象极深。7
日,斯科特给了他更多的上场时间,让他凭借身高优势,遏制“双塔”,多抢篮板,进攻时让
他尽量牵制对手。有了第一场的热身和激情,穆托姆博的第二场亮相似乎注定比第一场更好。人
们看到,他不仅得分了(4分),而且篮板和盖帽分别增加到4个和3个。基德的风头盖过了邓肯,
除了得分以外,主要是罚球。他8次罚球6次命中,其中在终场前1分多钟内就有6次罚球机会,结
果6罚5中。当时,双方比分接近,马刺又追得很紧,如果基德的罚球也像邓肯那么糟,网队也
就完了。但是,邓肯的不足恰好是基德的长处,本场比赛也不例外。
\chapter{谁是最有价值球员}
\echapter{Who is Most Valuable Player}
基德得势,网队胜利,穆托姆博同样功不可没。

\addcontentsline{toc}{chapter}{参考文献}
\addcontentsline{toe}{chapter}{Contents}
\nocite{*}
\begin{thebibliography}{99}
 \bibitem{njreds}\TeX{}中实现中英文的目录. \url{http://blog.csdn.net/reds/article/details/180943}
 \bibitem{chenshuo}\LaTeX{} 中插入中英双语目录. \url{http://blog.csdn.net/solstice/article/details/1589348}
\end{thebibliography}

\end{document}